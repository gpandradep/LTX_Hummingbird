% Based on reedthesis

%%
%% Preamble
%%
% \documentclass{<something>} must begin each LaTeX document
\documentclass[12pt,twoside]{reedthesis}
% Packages are extensions to the basic LaTeX functions. Whatever you
% want to typeset, there is probably a package out there for it.
% Chemistry (chemtex), screenplays, you name it.
% Check out CTAN to see: http://www.ctan.org/
%%
\usepackage[normalem]{ulem}
\usepackage[spanish,es-tabla]{babel}
\usepackage{setspace}\doublespacing
\usepackage{float}
\usepackage{graphicx,latexsym}
\usepackage{amsmath}
\usepackage{amssymb,amsthm}
\usepackage{longtable,booktabs,setspace}
\usepackage[hyphens]{url}
% Added by CII
\usepackage{lmodern}
\usepackage{float}
\floatplacement{figure}{H}
% End of CII addition
\usepackage{rotating}
\usepackage[hidelinks]{hyperref}
\usepackage[bottom, flushmargin]{footmisc}

% Spanish
\ifxetex
  \usepackage{polyglossia}
  \setmainlanguage{spanish}
 
  % Tabla en lugar de cuadro
  \addto\captionsspanish{%
  \renewcommand{\tablename}{Tabla}%
  \renewcommand{\listtablename}{Índice de tablas}%
  }
 
\else
  \usepackage[spanish,es-tabla]{babel}
  % Para los acentos (xelatex no lo necesita)
  \usepackage[utf8]{inputenc} 
  \usepackage[T1]{fontenc}
  \usepackage{lmodern}
\fi

% Next line commented out by CII
\usepackage{natbib}
% Comment out the natbib line above and uncomment the following two lines to use the new
% biblatex-chicago style, for Chicago A. Also make some changes at the end where the
% bibliography is included.
%\usepackage{biblatex-chicago}
%\bibliography{thesis}

\renewcommand{\baselinestretch}{1.5}
% Added by CII (Thanks, Hadley!)
% Use ref for internal links
\renewcommand{\hyperref}[2][???]{\autoref{#1}}
\def\chapterautorefname{Capítulo}
\def\sectionautorefname{Sección}
\def\subsectionautorefname{Subsección}
% End of CII addition


% \usepackage{times} % other fonts are available like times, bookman, charter, palatino
% Syntax highlighting #22

% To pass between YAML and LaTeX the dollar signs are added by CII
\title{¿Qué hay detras de la co-ocurrencia de mamíferos carnívoros?: un enfoque de fototrampeo y modelos jerárquicos}
\author{Gabriel Patricio Andrade Ponce}
\date{2020-03-05}
\advisor{Salvador Mandunajo Rodríguez}
\newcommand{\institution}{}
\newcommand{\type}{Doctorado en Ciencias}
%If you have two advisors for some reason, you can use the following
% Uncommented out by CII
% End of CII addition

% Added by CII
%%% Copied from knitr
%% maxwidth is the original width if it's less than linewidth
%% otherwise use linewidth (to make sure the graphics do not exceed the margin)

\renewcommand{\contentsname}{Índice general}
% End of CII addition

\setlength{\parskip}{0pt}

% Added by CII

\providecommand{\tightlist}{%
  \setlength{\itemsep}{0pt}\setlength{\parskip}{0pt}}

\Acknowledgements{
Under construction.
}

\Resumen{
En construcción

Under construction
}

\Dedication{
Pensando a quien dedicarla
}


\Abstract{

}

% End of CII addition
%%
%% End Preamble
%%
%
\begin{document}

% Everything below added by CII
% % \maketitle
%
\frontmatter % this stuff will be roman-numbered
\pagestyle{empty} % this removes page numbers from the frontmatter
\begin{flushleft}
  
  \medskip
  \medskip
  \includegraphics[width=0.60\textwidth]{Inecol.jpeg}\\\vspace {3.cm} 
  {\Large{\bf ¿Qué hay detras de la co-ocurrencia de mamíferos carnívoros?: un enfoque de fototrampeo y modelos jerárquicos}}\\\vspace{3.cm} 
  TESIS QUE PRESENTA \textit{\textbf{Gabriel Patricio Andrade Ponce}}\\
  PARA OBTENER EL GRADO DE \textit{\textbf{Doctorado en Ciencias}}\\\vspace{2.cm} 
  \medskip
  \medskip
  
  Xalapa, Veracrúz, México 2020-03-05
  
-------------------------------------------------  
\end{flushleft}
  \begin{dedication}
    Pensando a quien dedicarla
  \end{dedication}
  \begin{acknowledgements}
    Under construction.
  \end{acknowledgements}
  \begin{resumen}
    En construcción
    
    Under construction
  \end{resumen}


  \setcounter{tocdepth}{2}
  \tableofcontents

  \listoftables

  \listoffigures

\mainmatter % here the regular arabic numbering starts
\pagestyle{fancyplain} % turns page numbering back on

\hypertarget{introducciuxf3n-y-marco-teuxf3rico}{%
\chapter{Introducción y marco teórico}\label{introducciuxf3n-y-marco-teuxf3rico}}

\hypertarget{introducciuxf3n-general}{%
\section{Introducción general}\label{introducciuxf3n-general}}

El análisis de los patrones de co-ocurrencia de especies a partir de datos de presencia/ausencia, ha sido un tópico fundamental para el desarrollo de hipótesis y conceptos ecológicos relacionados con cómo se estructuran las comunidades y sus interacciones ecológicas
(Connor \& Simberloff, 1979; Diamond, 1975; Gotelli, 1999; HilleRisLambers, Adler, Harpole, Levine, \& Mayfield, 2012). La importancia de este tipo de estudios nace a partir del trabajo de Diamond (1975) sobre las \emph{reglas de ensamblaje} o \emph{assembly rules}, el cual inicia un debate sobre los procesos que subyacen a los patrones de co-ocurrencia de las especies. En términos generales Diamond (1975) propone a la competencia interespecífica como el principal mecanismo mediante el cual se estructuran de las comunidades, ya que especies que compiten no pueden ocupar el mismo espacio, dando como resultado un patrón de distribución excluyente. En años posteriores estas ideas fueron controvertidas por Connor \& Simberloff (1979), argumentando la existencia de los mismos patrones de co-ocurrencia en ausencia de la exclusión competitiva. Esta discusión llevó al desarrollo de modelos nulos para comparar los patrones observados de asociación de especies con lo esperado bajo un patrón aleatorio (Gotelli, 1999; Gotelli \& Ulrich, 2010), y de esta manera determinar si los patrones observados se deben a procesos estocásticos (neutrales) o determinísticos (no-neutrales) (Gotelli \& McCabe, 2002).

En años posteriores la disponibilidad de información ha generado un renovado interés por desarrollar nuevos algoritmos y modelos correlativos para inferir asociaciones entre especies, tanto a nivel de comunidad como de pares de especies (Bruelheide, 2000; Gotelli \& Ulrich, 2010; D. I. MacKenzie et al., 2018a; Veech, 2013). La mayoría de estas metodologías tienen como objetivo identificar y medir el grado de asociación que existe entre dos o más especies, y a su vez si este responde o no a un patrón aleatorio (Veech, 2006), es decir, si la co-ocurrencia de especies es más o menos probable de lo esperado por el azar.

Bajo esta aproximación es posible identificar dos tipos de asociaciones entre especies: de forma simétrica (+,+;-,-) como la facilitación y competencia, y de forma asimétrica (+,-) como las interacciones tróficas. En el caso de las asociaciones simétricas su interpretación bajo un enfoque espacial es relativamente sencilla, por ejemplo las asociaciones negativas (-,-) pueden ser un reflejo de segregación espacial dada por interacciones de competencia o de los requerimientos ambientales diferenciales de cada una de las especies. Por el contrario, asociaciones asimétricas presentan una dinámica espacial compleja, y por lo tanto su interpretación no es sencilla. Por ejemplo, interacciones de depredación pueden resultar en un patrón dinámico de co-ocurrencia donde el depredador y la presa se distribuyen con el fin de optimizar el forrajeo y la supervivencia respectivamente (Hammond, Luttbeg, Brodin, \& Sih, 2012), y de esta manera generar un patrón espacial de co-ocurrencia (positivo) o de segregación (negativo), el cual dependerá tanto de características de historia de vida de las especies, como la capacidad de dispersión de la especie presa (Clare, Linden, Anderson, \& MacFarland, 2016).

En el caso de los mamíferos, el creciente uso de cámaras trampa para el monitoreo de fauna ha generado gran cantidad de información de ocurrencias de especies alrededor del mundo (Mandujano, 2019; McCallum, 2013; Steenweg et al., 2017), lo cual se ha derivado en el creciente número de publicaciones sobre la composición de las comunidades de mamíferos (Ahumada et al., 2011; Rovero, Zimmermann, Berzi, \& Meek, 2013; Rowcliffe \& Carbone, 2008), y la asociación de especies o grupos de interés, como los mamíferos del orden carnívora (Davis et al., 2018; Rich et al., 2017). Particularmente, existe un interés en el estudio de los patrones de ocurrencia de los mamíferos carnívoros, no solo debido a que son un grupo carismático y relevante en términos de conservación (Jiménez et al., 2019; Minin et al., 2016; Ritchie \& Johnson, 2009; Schipper et al., 2009; Schipper, Hoffmann, Duckworth, \& Conroy, 2008; Sollmann, Mohamed, \& Kelly, 2013), sino también porque su historia evolutiva está fuertemente ligada a procesos de competencia (Macarthur \& Levins, 1967), y a que las especies que lo integran tienen una gran incidencia en el funcionamiento de las redes tróficas y el flujo de energía en los ecosistemas (Ritchie \& Johnson, 2009; Saggiomo, Picone, Esattore, \& Sommese, 2017).

En los mamíferos carnívoros el peso corporal ha sido identificado como uno de los principales factores que regula e influye las dinámicas
de interacción intragremio (Hutchinson, 1959; Wilson, 1975), de manera que especies de mayor tamaño y peso corporal segregan espacialmente a especies de menor tamaño. Dicho patrón se ha registrado en estudios a nivel global , donde especies de carnívoros co-ocurren menos de lo esperado con especies con un peso mayor a 15kg (Davis et al., 2018). No obstante, otros estudios a gran escala no registran dicho patrón en presencia de un depredador de gran porte (eg. Jaguar)(Santos et al., 2019). Este tipo de resultados o predicciones contrarias son comunes en estudios de co-ocurrencia (Davis et al., 2018; Gompper, Lesmeister, Ray, Malcolm, \& Kays, 2016; Lesmeister, Nielsen, Schauber, \& Hellgren, 2015; Robinson, Bustos, \& Roemer, 2014; Santos et al., 2019), lo que en parte se debe a que los patrones de co-ocurrencia, así como los mecanismos de segregación o asociación de las especies, también son sensibles a factores como la relación filogenética de las especies evaluadas, la estructura del hábitat, las relaciones tróficas, perturbaciones antrópicas, estrategias comportamentales y dinámicas climáticas (Amarasekare, 2003; Farris, Kelly, Karpanty, \& Ratelolahy, 2016; Green et al., 2018; Lonsinger, Gese, Bailey, \& Waits, 2017; Nagy-Reis, Nichols, Chiarello, Ribeiro, \& Setz, 2017; Robinson et al., 2014; Swanson et al., 2014; Zapata-Ríos \& Branch, 2018) e incluso a la escala de análisis y la probabilidad de detección de las especies.

Pese a la creciente cantidad de información no existe un consenso general sobre cuáles son los mecanismos o variables determinantes en la dinámica de co-ocurrencia de especies de carnívoros, ni el papel del diseño de muestreo en la detección y posterior inferencia de los patrones de co-ocurrencia. Debido a esto, el objetivo de la presente tesis es evaluar los patrones espacio-temporales de co-ocurrencia de mamíferos carnívoros, con énfasis en el uso de fototrampeo y modelos jerárquicos. Para ello, el primer objetivo especifico consiste en 1) identificar los factores que influyen en los patrones de co-ocurrencia de mamíferos carnívoros a partir de un meta-análisis de los resultados publicados en la literatura; posteriormente en el segundo objetivo se 2) analizarán los patrones de co-ocurrencia de dos especies de carnívoros y su presa potencial en en la selva seca de la región de la cañada de la Reserva de la Biosfera de Tehuacan-Cuicatlán, México. Como tercer objetivo se busca 3) evaluar el efecto del diseño de muestreo en los patrones de co-ocurrencia y su interpretación por medio de simulaciones de datos. Finalmente como cuarto objetivo se busca 4) Analizar la dinámica temporal de los patrones espaciales de co-ocurrencia de la comunidad de mamíferos, partir de modelos de ocupación de multi-especie para varias temporadas.

\hypertarget{marco-teuxf3rico}{%
\section{Marco teórico}\label{marco-teuxf3rico}}

\hypertarget{competencia-interespecifica}{%
\subsection{Competencia interespecifica}\label{competencia-interespecifica}}

Históricamente la competencia interespecifica ha sido identificada como uno de los principales mecanismos que influye en la estructura de las comunidades de especies (Diamond, 1975). Particularmente, los mamíferos carnívoros han sido utilizados como un grupo modelo, ya que la competencia ha sido un motor fundamental de cambio en su morfología, comportamiento y distribución (Donadio \& Buskirk, 2006; Tannerfeldt, Elmhagen, \& Angerbjörn, 2002).

Una interacción de competencia se da cuando dos o más especies que coexisten en simpatría presentan similitudes en su morfología, uso de recursos y/o nicho ecológico {[}Hardin (1960); (Linnell \& Strand, 2000). Dicha relación puede darse de dos maneras; el primer caso es la competencia de explotación (\emph{exploitative} en inglés), en donde dos especies utilizan el mismo recurso, pero una de ellas lo hace de manera más eficiente, la cual generalmente es un depredador especialista (Case \& Gilpin, 1974; Linnell \& Strand, 2000). El segundo caso es la competencia de interferencia, en la cual un depredador menos eficiente interfiere directamente sobre el otro competidor, lo cual resulta en el desplazamiento, kleptoparasitismo o incluso la depredación del competidor más pequeño o menos numeroso (Case \& Gilpin, 1974; Creel, 2001; Palomares, Caro, Byers, \& Holt, 1999; Polis, Myers, \& Holt, 1989). En algunos casos la depredación intragremios puede llegar a ser una de las principales causas de mortalidad en algunas especies de mamíferos,e incluso ser una amenaza importante para algunas especies bajo alguna categoría de conservación (Donadio \& Buskirk, 2006; Fleming et al., 2017; Palomares et al., 1999; White \& Garrott, 1997).

La forma e intensidad de las interacciones de competencia en carnívoros están dadas principalmente por características de su historia de vida, como el peso corporal o la dieta (Donadio \& Buskirk, 2006; Palomares et al., 1999). Diversos estudios sugieren que la intensidad de la interacción aumenta en especies con un mayor solape de dietas, ya que hábitos similares de forrajeo aumentan la probabilidad de un encuentro agonistico (Polis et al., 1989), así mismo temporadas o hábitats con escasez de recursos parecen incrementan la probabilidad de encuentros mortales (Cypher \& Spencer, 1998; Palomares et al., 1999). No obstante, también existe una relación entre la probabilidad de encuentros letales y el peso corporal de las especies, de manera que cuando un depredador posee de 2 a 5.4 veces la masa del otro competidor, su encuentro tiene la probabilidad máxima de resultar en la muerte de la especie más pequeña (Donadio \& Buskirk, 2006).

Debido al riesgo que supone un encuentro letal, las especies de menor tamaño presentan diversos mecanismos que les permiten reducir el solape ecológico con especies de mayor tamaño, ya sea por medio de segregación espacial, temporal o cambio en la dieta (Donadio \& Buskirk, 2006; Kozlowski, Gese, \& Arjo, 2012; Patterson, Willig, \& Stevens, 2003; Santos et al., 2019). La influencia de la competencia sobre la dinámica poblacional de las especies interactuantes, ha llevado al desarrollo modelos teóricos para predecir el comportamiento de especies interactuantes (Holt \& Polis, 1997). La hipótesis de depredación intragremios (IGP), predice que la relación de las especies de depredadores está dada por la abundancia del recurso compartido, en donde zonas o hábitats con mayor cantidad de presas serán dominadas por el depredador más grande, mientras que el meso predador será relegado a zonas de menor riqueza o abundancia de presas (Holt \& Polis, 1997; Robinson et al., 2014). De tal manera se espera que la dinámica de la competencia de ambas especies estará dada por la distribución y la abundancia espacio temporal de las presas, así como la vulnerabilidad de presas y meso predadores frente al depredador tope (Amarasekare, 2003, 2008; Capone, Carfora, De Luca, \& Torcicollo, 2018; Lesmeister et al., 2015; Moehrenschlager, List, \& Macdonald, 2007; Verdy \& Amarasekare, 2010). Pese a ello, no siempre se cumplen dichos supuestos de segregación, y en algunos casos mecanismos comportamentales como la vigilancia o la variación en la actividad diaria permiten la coexistencia de especies con un nicho similar en un mismo sitio (Bischof, Ali, Kabir, Hameed, \& Nawaz, 2014; Moehrenschlager et al., 2007; Santos et al., 2019).

\hypertarget{escala-de-anuxe1lisis-y-detecciuxf3n-imperfecta}{%
\subsection{Escala de análisis y detección imperfecta}\label{escala-de-anuxe1lisis-y-detecciuxf3n-imperfecta}}

Otros factores fundamentales que influyen sobre la interpretación de las asociaciones espaciales de las especies son la escala de análisis y la detección imperfecta. Por un lado, la escala de análisis es un factor que influye directamente sobre la interpretación y patrones de co-ocurrencia de las especies (Araújo \& Rozenfeld, 2014). Asociaciones antagónicas como la competencia y depredación (-,-;+,-) son particularmente susceptibles a la escala de análisis, ya que a medida que se aumenta la escala se diluyen los efectos de segregación y por lo tanto cambian los patrones de co-ocurrencia (Araújo \& Rozenfeld, 2014; Godsoe, Murray, \& Plank, 2015; Hammond et al., 2012). Por ejemplo, los efectos de la competencia sobre especies simpátricas son difíciles de identificar a una escala geográfica, ya que la coexistencia a nivel regional genera que los efectos de la exclusión por competencia no se manifiesten a una escala más grande de análisis (Godsoe et al., 2015)

De hecho, de acuerdo con las teorías de nicho ecológico y la hipótesis de ruido Eltoniano, la relevancia de las interacciones es inversa a la escala de análisis, por lo cual las interacciones ecológicas actúan principalmente a escala local y pierden relevancia en una escala geográfica, ya que a esta escala están subyugadas a variables macro climáticas y solo añaden efectos aleatorios a la distribución de las especies(Soberón \& Nakamura, 2009). No obstante, la mayoría de la evidencia a favor de la hipótesis de ruido Eltoniano proviene de estudios de simulaciones (Araújo \& Luoto, 2007) y existen casos particulares que sugieren evidencia de segregación espacial a nivel geográfico por medio de datos empíricos (Newsome \& Ripple, 2015). Adicionalmente, otros autores sugieren que a escala geográfica los resultados de las interacciones biológicas son enmascarados por la escala temporal a la que generalmente se realizan los análisis (décadas), y que interacciones de tipo antagónicas pueden evidenciarse a una escala temporal mucho mayor (Yackulic, 2017).

En el caso de mamíferos carnívoros, la mayoría de los estudios con fototrampeo se enmarcan en una escala a nivel local o nivel de paisaje (Davis et al., 2018), sin embargo, algunas especies de meso-depredadores y gran parte de las presas se mueven y seleccionan el hábitat y responden a la presencia de competidores o depredadores a una escala espacial fina (Bischof et al., 2014; Dröge, Creel, Becker, \& M'soka, 2017; Swanson, Arnold, Kosmala, Forester, \& Packer, 2016), lo cual también genera un reto al momento de interpretar los patrones de co-ocurrencia incluso a una escala local.

Por otro lado, la detección imperfecta también es un factor que puede generar conclusiones erróneas respecto a las asociaciones de especies, particularmente para datos de ocurrencias (MacKenzie et al., 2002, 2006), los cuales son susceptibles a errores de falsos negativos. Los falsos negativos se generan cuando la probabilidad de detectar un individuo es menor a uno, y debido a diferentes circunstancias, como el método, el tipo de hábitat, las condiciones climáticas, entre otros, una especie no es detectada, dado que realmente se encuentre en determinado sitio (Guillera‐Arroita, 2017). Se ha demostrado que no tener en cuenta la variación en la detección de especies conduce a sesgos en en la distribución espacial y una menor precisión de los parámetros estimados (Kéry \& Royle, 2015; D. I. MacKenzie et al., 2018b), y por consiguiente en sesgos cuando se consideran las relaciones espaciales de dos o más especies (MacKenzie, Bailey, \& Nichols, 2004; D. I. MacKenzie et al., 2018a).

Existen técnicas estadísticas que permiten lidiar con la variación en la detección de las especies, como los modelos jerárquicos (Kéry \& Royle, 2015), que a su vez permiten tener en cuenta otros factores bióticos, ambientales y antrópicos que puedan influir sobre los patrones de co-ocurrencia de las especies. Sin embargo, gran parte del efecto de la escala de análisis, recae sobre los protocolos empleados (posición y arreglo espacial de las cámaras trampa) y la interpretación de la variable de estado (Steenweg, Hebblewhite, Whittington, Lukacs, \& McKelvey, 2018). Esta discusión se ha abordado principalmente para estudios especie-específicos (Guillera‐Arroita \& Lahoz‐Monfort, 2012; Steenweg et al., 2018; Steenweg, Hebblewhite, Whittington, \& McKelvey, 2019), pero poco se ha discutido del efecto del arreglo en la detección de de dos o más especies y sus correspondientes asociaciones (Rich et al., 2019). Pese a ello, análisis realizados en otros grupos de estudio sugieren que los efectos de la escala sobre los patrones de co-ocurrencia de especies y su interpretación es dependiente del sistema (Thurman, Barner, Garcia, \& Chestnut, 2019), y en ese sentido del hábitat y su productividad, la historia de vida de las especies y el número de especies involucradas (Davis et al., 2018).

\hypertarget{bibliografuxeda}{%
\section{Bibliografía}\label{bibliografuxeda}}

\hypertarget{refs}{}
\leavevmode\hypertarget{ref-ahumada_community_2011}{}%
Ahumada, J. A., Silva, C. E., Gajapersad, K., Hallam, C., Hurtado, J., Martin, E., \ldots{} Rovero, F. (2011). Community structure and diversity of tropical forest mammals: Data from a global camera trap network. \emph{Philosophical Transactions of the Royal Society B: Biological Sciences}, \emph{366}(1578), 2703--2711.

\leavevmode\hypertarget{ref-amarasekare_competitive_2003}{}%
Amarasekare, P. (2003). Competitive coexistence in spatially structured environments: A synthesis. \emph{Ecology Letters}, \emph{6}(12), 1109--1122. \url{http://doi.org/10.1046/j.1461-0248.2003.00530.x}

\leavevmode\hypertarget{ref-amarasekare_coexistence_2008}{}%
Amarasekare, P. (2008). Coexistence of Intraguild Predators and Prey in Resource-Rich Environments. \emph{Ecology}, \emph{89}(10), 2786--2797. \url{http://doi.org/10.1890/07-1508.1}

\leavevmode\hypertarget{ref-araujo_importance_2007}{}%
Araújo, M. B., \& Luoto, M. (2007). The importance of biotic interactions for modelling species distributions under climate change. \emph{Global Ecology and Biogeography}, \emph{16}(6), 743--753. \url{http://doi.org/10.1111/j.1466-8238.2007.00359.x}

\leavevmode\hypertarget{ref-araujo_geographic_2014}{}%
Araújo, M. B., \& Rozenfeld, A. (2014). The geographic scaling of biotic interactions. \emph{Ecography}, \emph{37}(5), 406--415. \url{http://doi.org/10.1111/j.1600-0587.2013.00643.x}

\leavevmode\hypertarget{ref-bischof_being_2014}{}%
Bischof, R., Ali, H., Kabir, M., Hameed, S., \& Nawaz, M. A. (2014). Being the underdog: An elusive small carnivore uses space with prey and time without enemies. \emph{Journal of Zoology}, \emph{293}(1), 40--48. \url{http://doi.org/10.1111/jzo.12100}

\leavevmode\hypertarget{ref-bruelheide_new_2000}{}%
Bruelheide, H. (2000). A new measure of fidelity and its application to defining species groups. \emph{Journal of Vegetation Science}, \emph{11}(2), 167--178.

\leavevmode\hypertarget{ref-capone_dynamics_2018}{}%
Capone, F., Carfora, M. F., De Luca, R., \& Torcicollo, I. (2018). On the dynamics of an intraguild predator--prey model. \emph{Mathematics and Computers in Simulation}, \emph{149}, 17--31. \url{http://doi.org/10.1016/j.matcom.2018.01.004}

\leavevmode\hypertarget{ref-case_interference_1974}{}%
Case, T. J., \& Gilpin, M. E. (1974). Interference Competition and Niche Theory. \emph{Proceedings of the National Academy of Sciences of the United States of America}, \emph{71}(8), 3073--3077. Retrieved from \url{https://www.ncbi.nlm.nih.gov/pmc/articles/PMC388623/}

\leavevmode\hypertarget{ref-clare_antipredator_2016}{}%
Clare, J. D. J., Linden, D. W., Anderson, E. M., \& MacFarland, D. M. (2016). Do the antipredator strategies of shared prey mediate intraguild predation and mesopredator suppression? \emph{Ecology and Evolution}, \emph{6}(12), 3884--3897. \url{http://doi.org/10.1002/ece3.2170}

\leavevmode\hypertarget{ref-connor_assembly_1979}{}%
Connor, E. F., \& Simberloff, D. (1979). The Assembly of Species Communities: Chance or Competition? \emph{Ecology}, \emph{60}(6), 1132--1140. \url{http://doi.org/10.2307/1936961}

\leavevmode\hypertarget{ref-creel_four_2001}{}%
Creel, S. (2001). Four factors modifying the effect of competition on carnivore population dynamics as illustrated by African wild dogs. \emph{Conservation Biology}, \emph{15}(1), 271--274.

\leavevmode\hypertarget{ref-cypher_competitive_1998}{}%
Cypher, B. L., \& Spencer, K. A. (1998). Competitive Interactions between Coyotes and San Joaquin Kit Foxes. \emph{Journal of Mammalogy}, \emph{79}(1), 204--214. \url{http://doi.org/10.2307/1382855}

\leavevmode\hypertarget{ref-davis_ecological_2018}{}%
Davis, C. L., Rich, L. N., Farris, Z. J., Kelly, M. J., Bitetti, M. S. D., Blanco, Y. D., \ldots{} Miller, D. A. W. (2018). Ecological correlates of the spatial co-occurrence of sympatric mammalian carnivores worldwide. \emph{Ecology Letters}, \emph{21}, 1401--1412. \url{http://doi.org/10.1111/ele.13124}

\leavevmode\hypertarget{ref-diamond_assembly_1975}{}%
Diamond, J. M. (1975). Assembly of species communities. In M. L. Cody \& J. M. Diamond (Eds.), \emph{Ecology and evolution of communities} (pp. 342--444). Boston: Harvard University Press.

\leavevmode\hypertarget{ref-donadio_diet_2006}{}%
Donadio, E., \& Buskirk, S. W. (2006). Diet, morphology, and interspecific killing in Carnivora. \emph{The American Naturalist}, \emph{167}(4), 524--536.

\leavevmode\hypertarget{ref-droge_spatial_2017}{}%
Dröge, E., Creel, S., Becker, M. S., \& M'soka, J. (2017). Spatial and temporal avoidance of risk within a large carnivore guild. \emph{Ecology and Evolution}, \emph{7}(1), 189--199. \url{http://doi.org/10.1002/ece3.2616}

\leavevmode\hypertarget{ref-farris_patterns_2016}{}%
Farris, Z. J., Kelly, M. J., Karpanty, S., \& Ratelolahy, F. (2016). Patterns of spatial co-occurrence among native and exotic carnivores in north-eastern Madagascar. \emph{Animal Conservation}, \emph{19}(2), 189--198. \url{http://doi.org/10.1111/acv.12233}

\leavevmode\hypertarget{ref-fleming_roles_2017}{}%
Fleming, P. J. S., Nolan, H., Jackson, S. M., Ballard, G.-A., Bengsen, A., Brown, W. Y., \ldots{} Sparkes, J. (2017). Roles for the Canidae in food webs reviewed: Where do they fit? \emph{Food Webs}, \emph{12}, 14--34. \url{http://doi.org/10.1016/j.fooweb.2017.03.001}

\leavevmode\hypertarget{ref-godsoe_effect_2015}{}%
Godsoe, W., Murray, R., \& Plank, M. J. (2015). The effect of competition on species' distributions depends on coexistence, rather than scale alone. \emph{Ecography}, \emph{38}(11), 1071--1079. \url{http://doi.org/10.1111/ecog.01134}

\leavevmode\hypertarget{ref-gompper_differential_2016}{}%
Gompper, M. E., Lesmeister, D. B., Ray, J. C., Malcolm, J. R., \& Kays, R. (2016). Differential Habitat Use or Intraguild Interactions: What Structures a Carnivore Community? \emph{PLOS ONE}, \emph{11}(1), e0146055. \url{http://doi.org/10.1371/journal.pone.0146055}

\leavevmode\hypertarget{ref-gotelli_how_1999}{}%
Gotelli, N. J. (1999). How Do Communities Come Together? \emph{Science}, \emph{286}(5445), 1684--1685. \url{http://doi.org/10.1126/science.286.5445.1684a}

\leavevmode\hypertarget{ref-gotelli_species_2002}{}%
Gotelli, N. J., \& McCabe, D. J. (2002). Species co‐occurrence: A meta‐analysis of JM Diamond's assembly rules model. \emph{Ecology}, \emph{83}(8), 2091--2096.

\leavevmode\hypertarget{ref-gotelli_empirical_2010}{}%
Gotelli, N. J., \& Ulrich, W. (2010). The empirical Bayes approach as a tool to identify non-random species associations. \emph{Oecologia}, \emph{162}(2), 463--477.

\leavevmode\hypertarget{ref-green_dynamic_2018}{}%
Green, D. S., Matthews, S. M., Swiers, R. C., Callas, R. L., Yaeger, J. S., Farber, S. L., \ldots{} Powell, R. A. (2018). Dynamic occupancy modelling reveals a hierarchy of competition among fishers, grey foxes and ringtails. \emph{Journal of Animal Ecology}, \emph{87}(3), 813--824. \url{http://doi.org/10.1111/1365-2656.12791}

\leavevmode\hypertarget{ref-guilleraarroita_modelling_2017}{}%
Guillera‐Arroita, G. (2017). Modelling of species distributions, range dynamics and communities under imperfect detection: Advances, challenges and opportunities. \emph{Ecography}, \emph{40}(2), 281--295.

\leavevmode\hypertarget{ref-guilleraarroita_designing_2012}{}%
Guillera‐Arroita, G., \& Lahoz‐Monfort, J. J. (2012). Designing studies to detect differences in species occupancy: Power analysis under imperfect detection. \emph{Methods in Ecology and Evolution}, \emph{3}(5), 860--869.

\leavevmode\hypertarget{ref-hammond_spatial_2012}{}%
Hammond, J. I., Luttbeg, B., Brodin, T., \& Sih, A. (2012). Spatial scale influences the outcome of the predator--prey space race between tadpoles and predatory dragonflies. \emph{Functional Ecology}, \emph{26}(2), 522--531.

\leavevmode\hypertarget{ref-hardin_competitive_1960}{}%
Hardin, G. (1960). The Competitive Exclusion Principle. \emph{Science}, \emph{131}(3409), 1292--1297. \url{http://doi.org/10.1126/science.131.3409.1292}

\leavevmode\hypertarget{ref-hillerislambers_rethinking_2012}{}%
HilleRisLambers, J., Adler, P. B., Harpole, W. S., Levine, J. M., \& Mayfield, M. M. (2012). Rethinking Community Assembly through the Lens of Coexistence Theory. \emph{Annual Review of Ecology, Evolution, and Systematics}, \emph{43}(1), 227--248. \url{http://doi.org/10.1146/annurev-ecolsys-110411-160411}

\leavevmode\hypertarget{ref-holt_theoretical_1997}{}%
Holt, R. D., \& Polis, G. A. (1997). A Theoretical Framework for Intraguild Predation. \emph{The American Naturalist}, \emph{149}(4), 745--764. Retrieved from \url{http://www.jstor.org/stable/2463547}

\leavevmode\hypertarget{ref-hutchinson_homage_1959}{}%
Hutchinson, G. E. (1959). Homage to Santa Rosalia or Why Are There So Many Kinds of Animals? \emph{The American Naturalist}, \emph{93}(870), 145--159. \url{http://doi.org/10.1086/282070}

\leavevmode\hypertarget{ref-jimenez_restoring_2019}{}%
Jiménez, J., Nuñez-Arjona, J. C., Mougeot, F., Ferreras, P., González, L. M., García-Domínguez, F., \ldots{} Rueda, C. (2019). Restoring apex predators can reduce mesopredator abundances. \emph{Biological Conservation}, \emph{238}, 108234.

\leavevmode\hypertarget{ref-kery_applied_2015}{}%
Kéry, M., \& Royle, A. (2015). \emph{Applied Hierarchical Modeling in Ecology: Analysis of distribution, abundance and species richness in R and BUGS - 1st Edition} (Elsevier, Vol. 1). Retrieved from \url{https://www.elsevier.com/books/applied-hierarchical-modeling-in-ecology-analysis-of-distribution-abundance-and-species-richness-in-r-and-bugs/kery/978-0-12-801378-6}

\leavevmode\hypertarget{ref-kozlowski_effects_2012}{}%
Kozlowski, A. J., Gese, E. M., \& Arjo, W. M. (2012). Effects of Intraguild Predation: Evaluating Resource Competition between Two Canid Species with Apparent Niche Separation. \emph{International Journal of Ecology}, \emph{2012}, 1--12. \url{http://doi.org/10.1155/2012/629246}

\leavevmode\hypertarget{ref-lesmeister_spatial_2015}{}%
Lesmeister, D., Nielsen, C., Schauber, E., \& Hellgren, E. (2015). Spatial and Temporal Structure of a Mesocarnivore Guild in Midwestern North America. \emph{Wildlife Monographs}. \url{http://doi.org/10.1002/wmon.1015}

\leavevmode\hypertarget{ref-linnell_interference_2000}{}%
Linnell, J. D. C., \& Strand, O. (2000). Interference interactions, co-existence and conservation of mammalian carnivores. \emph{Diversity and Distributions}, \emph{6}(4), 169--176. \url{http://doi.org/10.1046/j.1472-4642.2000.00069.x}

\leavevmode\hypertarget{ref-lonsinger_roles_2017}{}%
Lonsinger, R. C., Gese, E. M., Bailey, L. L., \& Waits, L. P. (2017). The roles of habitat and intraguild predation by coyotes on the spatial dynamics of kit foxes. \emph{Ecosphere}, \emph{8}(3), 1--14. \url{http://doi.org/10.1002/ecs2.1749}

\leavevmode\hypertarget{ref-macarthur_limiting_1967}{}%
Macarthur, R., \& Levins, R. (1967). The Limiting Similarity, Convergence, and Divergence of Coexisting Species. \emph{The American Naturalist}, \emph{101}(921), 377--385. \url{http://doi.org/10.1086/282505}

\leavevmode\hypertarget{ref-mackenzie_investigating_2004}{}%
MacKenzie, D. I., Bailey, L. L., \& Nichols, J. D. (2004). Investigating species co-occurrence patterns when species are detected imperfectly. \emph{Journal of Animal Ecology}, \emph{73}(3), 546--555. \url{http://doi.org/10.1111/j.0021-8790.2004.00828.x}

\leavevmode\hypertarget{ref-mackenzie_estimating_2002-1}{}%
MacKenzie, D. I., Nichols, J. D., Lachman, G. B., Droege, S., Andrew Royle, J., \& Langtimm, C. A. (2002). Estimating site occupancy rates when detection probabilities are less than one. \emph{Ecology}, \emph{83}(8), 2248--2255. \url{http://doi.org/10.1890/0012-9658(2002)083\%5B2248:ESORWD\%5D2.0.CO;2}

\leavevmode\hypertarget{ref-mackenzie_chapter_2018-2}{}%
MacKenzie, D. I., Nichols, J. D., Royle, J. A., Pollock, K. H., Bailey, L. L., \& Hines, J. E. (2018a). Chapter 14 - Species Co-Occurrence. In D. I. MacKenzie, J. D. Nichols, J. A. Royle, K. H. Pollock, L. L. Bailey, \& J. E. Hines (Eds.), \emph{Occupancy Estimation and Modeling (Second Edition)} (Second Edition, pp. 509--556). Boston: Academic Press. \url{http://doi.org/10.1016/B978-0-12-407197-1.00019-3}

\leavevmode\hypertarget{ref-mackenzie_chapter_2018}{}%
MacKenzie, D. I., Nichols, J. D., Royle, J. A., Pollock, K. H., Bailey, L. L., \& Hines, J. E. (2018b). Chapter 4 - Basic Presence/Absence Situation. In D. I. MacKenzie, J. D. Nichols, J. A. Royle, K. H. Pollock, L. L. Bailey, \& J. E. Hines (Eds.), \emph{Occupancy Estimation and Modeling (Second Edition)} (pp. 115--215). Boston: Academic Press. \url{http://doi.org/10.1016/B978-0-12-407197-1.00006-5}

\leavevmode\hypertarget{ref-mackenzie_occupancy_2006}{}%
MacKenzie, D., Nichols, J., Royle, J., Pollock, K. A., Bailey, L. L., \& Hines, J. (2006). \emph{Occupancy Estimation and Modeling} (1st ed.). Academic Press. Retrieved from \url{https://www.elsevier.com/books/occupancy-estimation-and-modeling/mackenzie/978-0-12-088766-8}

\leavevmode\hypertarget{ref-mandujano_analysis_2019}{}%
Mandujano, S. (2019). Analysis and trends of photo-trapping in Mexico: Text mining in R. \emph{THERYA}, \emph{10}(1), 25. Retrieved from \url{http://www.revistas-conacyt.unam.mx/therya/index.php/THERYA/article/view/666}

\leavevmode\hypertarget{ref-mccallum_changing_2013}{}%
McCallum, J. (2013). Changing use of camera traps in mammalian field research: Habitats, taxa and study types. \emph{Mammal Review}, \emph{43}(3), 196--206.

\leavevmode\hypertarget{ref-minin_global_2016}{}%
Minin, E. D., Slotow, R., Hunter, L. T. B., Pouzols, F. M., Toivonen, T., Verburg, P. H., \ldots{} Moilanen, A. (2016). Global priorities for national carnivore conservation under land use change. \emph{Scientific Reports}, \emph{6}(1), 1--9. \url{http://doi.org/10.1038/srep23814}

\leavevmode\hypertarget{ref-moehrenschlager_escaping_2007}{}%
Moehrenschlager, A., List, R., \& Macdonald, D. W. (2007). Escaping Intraguild Predation: Mexican Kit Foxes Survive While Coyotes and Golden Eagles Kill Canadian Swift Foxes. \emph{Journal of Mammalogy}, \emph{88}(4), 1029--1039. \url{http://doi.org/10.1644/06-MAMM-A-159R.1}

\leavevmode\hypertarget{ref-nagy-reis_landscape_2017}{}%
Nagy-Reis, M. B., Nichols, J. D., Chiarello, A. G., Ribeiro, M. C., \& Setz, E. Z. F. (2017). Landscape Use and Co-Occurrence Patterns of Neotropical Spotted Cats. \emph{PLOS ONE}, \emph{12}(1), e0168441. \url{http://doi.org/10.1371/journal.pone.0168441}

\leavevmode\hypertarget{ref-newsome_continental_2015}{}%
Newsome, T. M., \& Ripple, W. J. (2015). A continental scale trophic cascade from wolves through coyotes to foxes. \emph{Journal of Animal Ecology}, \emph{84}(1), 49--59. \url{http://doi.org/10.1111/1365-2656.12258}

\leavevmode\hypertarget{ref-palomares_interspecific_1999}{}%
Palomares, F., Caro, T. M., Byers, A. E. J. A., \& Holt, R. D. (1999). Interspecific Killing among Mammalian Carnivores. \emph{The American Naturalist}, \emph{153}(5), 492--508. \url{http://doi.org/10.1086/303189}

\leavevmode\hypertarget{ref-patterson_trophic_2003}{}%
Patterson, B. D., Willig, M., \& Stevens, R. D. (2003). Trophic strategies, niche partitioning, and patterns of ecological organization. In Kunz \& M. B. Fenton (Eds.), \emph{Bat ecology} (pp. 536--579). Chicago, USA: University of Chicago Press.

\leavevmode\hypertarget{ref-polis_ecology_1989}{}%
Polis, G. A., Myers, C. A., \& Holt, R. D. (1989). The Ecology and Evolution of Intraguild Predation: Potential Competitors That Eat Each Other. \emph{Annual Review of Ecology and Systematics}, \emph{20}(1), 297--330. \url{http://doi.org/10.1146/annurev.es.20.110189.001501}

\leavevmode\hypertarget{ref-rich_assessing_2017}{}%
Rich, L. N., Davis, C. L., Farris, Z. J., Miller, D. A., Tucker, J. M., Hamel, S., \ldots{} Thapa, K. (2017). Assessing global patterns in mammalian carnivore occupancy and richness by integrating local camera trap surveys. \emph{Global Ecology and Biogeography}, \emph{26}(8), 918--929.

\leavevmode\hypertarget{ref-rich_sampling_2019}{}%
Rich, L. N., Miller, D. A., Muñoz, D. J., Robinson, H. S., McNutt, J. W., \& Kelly, M. J. (2019). Sampling design and analytical advances allow for simultaneous density estimation of seven sympatric carnivore species from camera trap data. \emph{Biological Conservation}, \emph{233}, 12--20.

\leavevmode\hypertarget{ref-ritchie_predator_2009}{}%
Ritchie, E. G., \& Johnson, C. N. (2009). Predator interactions, mesopredator release and biodiversity conservation. \emph{Ecology Letters}, \emph{12}(9), 982--998. \url{http://doi.org/10.1111/j.1461-0248.2009.01347.x}

\leavevmode\hypertarget{ref-robinson_application_2014}{}%
Robinson, Q. H., Bustos, D., \& Roemer, G. W. (2014). The application of occupancy modeling to evaluate intraguild predation in a model carnivore system. \emph{Ecology}, \emph{95}(11), 3112--3123. \url{http://doi.org/10.1890/13-1546.1}

\leavevmode\hypertarget{ref-rovero_which_2013}{}%
Rovero, F., Zimmermann, F., Berzi, D., \& Meek, P. (2013). "Which camera trap type and how many do I need?" A review of camera features and study designs for a range of wildlife research applications. \emph{Hystrix, the Italian Journal of Mammalogy}, \emph{24}(2), 148--156. \url{http://doi.org/10.4404/hystrix-24.2-8789}

\leavevmode\hypertarget{ref-rowcliffe_surveys_2008}{}%
Rowcliffe, J. M., \& Carbone, C. (2008). Surveys using camera traps: Are we looking to a brighter future? \emph{Animal Conservation}, \emph{11}(3), 185--186.

\leavevmode\hypertarget{ref-saggiomo_overview_2017}{}%
Saggiomo, L., Picone, F., Esattore, B., \& Sommese, A. (2017). An overview of understudied interaction types amongst large carnivores. \emph{Food Webs}, \emph{12}, 35--39. \url{http://doi.org/10.1016/j.fooweb.2017.01.001}

\leavevmode\hypertarget{ref-santos_prey_2019}{}%
Santos, F., Carbone, C., Wearn, O. R., Rowcliffe, J. M., Espinosa, S., Lima, M. G. M., \ldots{} Peres, C. A. (2019). Prey availability and temporal partitioning modulate felid coexistence in Neotropical forests. \emph{PLOS ONE}, \emph{14}(3), e0213671. \url{http://doi.org/10.1371/journal.pone.0213671}

\leavevmode\hypertarget{ref-schipper_small_2009}{}%
Schipper, J., Helgen, K. M., Belant, J., Gonzalez-Maya, J., Eizirik, E., \& Tsuchiya-Jerep, M. T. (2009). Small carnivores in the Americas: Reflections, future research and conservation priorities. \emph{Small Carnivore Conservation}.

\leavevmode\hypertarget{ref-schipper_2008_2008}{}%
Schipper, J., Hoffmann, M., Duckworth, J. W., \& Conroy, J. (2008). The 2008 IUCN red listings of the world's small carnivores, \emph{39}, 6.

\leavevmode\hypertarget{ref-soberon_niches_2009}{}%
Soberón, J., \& Nakamura, M. (2009). Niches and distributional areas: Concepts, methods, and assumptions. \emph{Proceedings of the National Academy of Sciences of the United States of America}, \emph{106 Suppl 2}, 19644--19650. \url{http://doi.org/10.1073/pnas.0901637106}

\leavevmode\hypertarget{ref-sollmann_camera_2013}{}%
Sollmann, R., Mohamed, A., \& Kelly, M. J. (2013). Camera trapping for the study and conservation of tropical carnivores. \emph{Raffles Bulletin of Zoology}, \emph{28}, 21--42.

\leavevmode\hypertarget{ref-steenweg_scalingup_2017}{}%
Steenweg, R., Hebblewhite, M., Kays, R., Ahumada, J., Fisher, J. T., Burton, C., \ldots{} Whittington, J. (2017). Scaling‐up camera traps: Monitoring the planet's biodiversity with networks of remote sensors. \emph{Frontiers in Ecology and the Environment}, \emph{15}(1), 26--34.

\leavevmode\hypertarget{ref-steenweg_sampling_2018}{}%
Steenweg, R., Hebblewhite, M., Whittington, J., Lukacs, P., \& McKelvey, K. (2018). Sampling scales define occupancy and underlying occupancy--abundance relationships in animals. \emph{Ecology}, \emph{99}(1), 172--183.

\leavevmode\hypertarget{ref-steenweg_speciesspecific_2019}{}%
Steenweg, R., Hebblewhite, M., Whittington, J., \& McKelvey, K. (2019). Species‐specific differences in detection and occupancy probabilities help drive ability to detect trends in occupancy. \emph{Ecosphere}, \emph{10}(4), e02639.

\leavevmode\hypertarget{ref-swanson_absence_2016}{}%
Swanson, A., Arnold, T., Kosmala, M., Forester, J., \& Packer, C. (2016). In the absence of a ``landscape of fear'': How lions, hyenas, and cheetahs coexist. \emph{Ecology and Evolution}, \emph{6}(23), 8534--8545.

\leavevmode\hypertarget{ref-swanson_cheetahs_2014}{}%
Swanson, A., Caro, T., Davies-Mostert, H., Mills, M. G. L., Macdonald, D. W., Borner, M., \ldots{} Packer, C. (2014). Cheetahs and wild dogs show contrasting patterns of suppression by lions. \emph{The Journal of Animal Ecology}, \emph{83}(6), 1418--1427. \url{http://doi.org/10.1111/1365-2656.12231}

\leavevmode\hypertarget{ref-tannerfeldt_exclusion_2002}{}%
Tannerfeldt, M., Elmhagen, B., \& Angerbjörn, A. (2002). Exclusion by interference competition? The relationship between red and arctic foxes. \emph{Oecologia}, \emph{132}(2), 213--220.

\leavevmode\hypertarget{ref-thurman_testing_2019}{}%
Thurman, L. L., Barner, A. K., Garcia, T. S., \& Chestnut, T. (2019). Testing the link between species interactions and species co occurrence in a trophic network. \emph{Ecography}, \emph{42}(10), 1658--1670.

\leavevmode\hypertarget{ref-veech_probabilitybased_2006}{}%
Veech, J. A. (2006). A probability‐based analysis of temporal and spatial co‐occurrence in grassland birds. \emph{Journal of Biogeography}, \emph{33}(12), 2145--2153.

\leavevmode\hypertarget{ref-veech_probabilistic_2013}{}%
Veech, J. A. (2013). A probabilistic model for analysing species co‐occurrence. \emph{Global Ecology and Biogeography}, \emph{22}(2), 252--260.

\leavevmode\hypertarget{ref-verdy_alternative_2010}{}%
Verdy, A., \& Amarasekare, P. (2010). Alternative stable states in communities with intraguild predation. \emph{Journal of Theoretical Biology}, \emph{262}(1), 116--128. \url{http://doi.org/10.1016/j.jtbi.2009.09.011}

\leavevmode\hypertarget{ref-white_factors_1997}{}%
White, P. J., \& Garrott, R. A. (1997). Factors regulating kit fox populations. \emph{Canadian Journal of Zoology}, \emph{75}(12), 1982--1988. \url{http://doi.org/10.1139/z97-830}

\leavevmode\hypertarget{ref-wilson_adequacy_1975}{}%
Wilson, D. S. (1975). The Adequacy of Body Size as a Niche Difference. \emph{The American Naturalist}, \emph{109}(970), 769--784. Retrieved from \url{https://www.jstor.org/stable/2459869}

\leavevmode\hypertarget{ref-yackulic_competitive_2017}{}%
Yackulic, C. B. (2017). Competitive exclusion over broad spatial extents is a slow process: Evidence and implications for species distribution modeling. \emph{Ecography}, \emph{40}(2), 305--313. \url{http://doi.org/10.1111/ecog.02836}

\leavevmode\hypertarget{ref-zapata-rios_mammalian_2018}{}%
Zapata-Ríos, G., \& Branch, L. C. (2018). Mammalian carnivore occupancy is inversely related to presence of domestic dogs in the high Andes of Ecuador. \emph{PLOS ONE}, \emph{13}(2), e0192346. \url{http://doi.org/10.1371/journal.pone.0192346}


% Index?

\end{document}
